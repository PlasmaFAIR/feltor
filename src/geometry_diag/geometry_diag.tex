%%%%%%%%%%%%%%%%%%%%%definitions%%%%%%%%%%%%%%%%%%%%%%%%%%%%%%%%%%%%%%%

%\documentclass[12pt]{article}
%\documentclass[12pt]{scrartcl}
\documentclass{hitec} % contained in texlive-latex-extra
\settextfraction{0.9} % indent text
\usepackage{csquotes}
\usepackage[hidelinks]{hyperref} % doi links are short and usefull?
\hypersetup{%
    colorlinks=true,
    linkcolor=blue,
    urlcolor=magenta
}
\urlstyle{rm}
\usepackage[english]{babel}
\usepackage{mathtools} % loads and extends amsmath
\usepackage{amssymb}
% packages not used
%\usepackage{graphicx}
%\usepackage{amsthm}
%\usepackage{subfig}
\usepackage{bm}
\usepackage{geometry}
\usepackage{longtable}
\usepackage{booktabs}
\usepackage{ragged2e} % maybe use \RaggedRight for tables and literature?
\usepackage[table]{xcolor} % for alternating colors
%\rowcolors{2}{gray!25}{white} %%% Use this line in front of longtable
\renewcommand\arraystretch{1.3}
\usepackage[most]{tcolorbox}
\usepackage{doi}
\usepackage[sort,square,numbers]{natbib}
\bibliographystyle{abbrvnat}
%%% reset bibliography distances %%%
\let\oldthebibliography\thebibliography
\let\endoldthebibliography\endthebibliography
\renewenvironment{thebibliography}[1]{
  \begin{oldthebibliography}{#1}
    \RaggedRight % remove if justification is desired
    \setlength{\itemsep}{0em}
    \setlength{\parskip}{0em}
}
{
  \end{oldthebibliography}
}
%%% --- %%%

\definecolor{light-gray}{gray}{0.95}
\newcommand{\code}[1]{\colorbox{light-gray}{\texttt{#1}}}
\newcommand{\eps}{\varepsilon}
\renewcommand{\d}{\mathrm{d}}
\renewcommand{\vec}[1]{{\boldsymbol{#1}}}
\newcommand{\dx}{\,\mathrm{d}x}
%\newcommand{\dA}{\,\mathrm{d}(x,y)}
%\newcommand{\dV}{\mathrm{d}^3{x}\,}
\newcommand{\dA}{\,\mathrm{dA}}
\newcommand{\dV}{\mathrm{dV}\,}

\newcommand{\Eins}{\mathbf{1}}

\newcommand{\ExB}{$\bm{E}\times\bm{B} \,$}
\newcommand{\GKI}{\int d^6 \bm{Z} \BSP}
\newcommand{\GKIV}{\int dv_{\|} d \mu d \theta \BSP}
\newcommand{\BSP}{B_{\|}^*}
\newcommand{\Abar}{\langle A_\parallel \rangle}
%Averages
\newcommand{\RA}[1]{\left \langle #1 \right \rangle} %Reynolds (flux-surface) average
\newcommand{\RF}[1]{\widetilde{#1}} %Reynolds fluctuation
\newcommand{\FA}[1]{\left[\left[ #1 \right]\right]} %Favre average
\newcommand{\FF}[1]{\widehat{#1}} %Favre fluctuation
\newcommand{\PA}[1]{\left \langle #1 \right\rangle_\varphi} %Phi average

%Vectors
\newcommand{\ahat}{\bm{\hat{a}}}
\newcommand{\bhat}{\bm{\hat{b}}}
\newcommand{\chat}{\bm{\hat{c}}}
\newcommand{\ehat}{\bm{\hat{e}}}
\newcommand{\bbar}{\overline{\bm{b}}}
\newcommand{\xhat}{\bm{\hat{x}}}
\newcommand{\yhat}{\bm{\hat{y}}}
\newcommand{\zhat}{\bm{\hat{z}}}

\newcommand{\Xbar}{\bar{\vec{X}}}
\newcommand{\phat}{\bm{\hat{\perp}}}
\newcommand{\that}{\bm{\hat{\theta}}}

\newcommand{\eI}{\bm{\hat{e}}_1}
\newcommand{\eII}{\bm{\hat{e}}_2}
\newcommand{\ud}{\mathrm{d}}

%Derivatives etc.
\newcommand{\pfrac}[2]{\frac{\partial#1}{\partial#2}}
\newcommand{\ffrac}[2]{\frac{\delta#1}{\delta#2}}
\newcommand{\fixd}[1]{\Big{\arrowvert}_{#1}}
\newcommand{\curl}[1]{\nabla \times #1}

\newcommand{\np}{\vec{\nabla}_{\perp}}
\newcommand{\npc}{\nabla_{\perp} \cdot }
\newcommand{\nc}{\vec\nabla\cdot}
\newcommand{\cn}{\cdot\vec\nabla}
\newcommand{\vn}{\vec{\nabla}}
\newcommand{\npar}{\nabla_\parallel}

\newcommand{\KB}{ {\vec K_{\vn B}}}
\newcommand{\KK}{ {\vec K_{\vn\times\bhat}}}

\newcommand{\GAI}{\Gamma_{1}^{\dagger}}
\newcommand{\GAII}{\Gamma_{1}^{\dagger -1}}
\newcommand{\T}{\mathrm{T}}
\newcommand{\Tp}{\mathcal T^+_{\Delta\varphi}}
\newcommand{\Tm}{\mathcal T^-_{\Delta\varphi}}
\newcommand{\Tpm}{\mathcal T^\pm_{\Delta\varphi}}
\newcommand{\Tdp}{\mathcal T^+_{\delta\varphi}}
\newcommand{\Tdm}{\mathcal T^-_{\delta\varphi}}
\newcommand{\Tdpm}{\mathcal T^\pm_{\delta\varphi}}
%%%%%%%%Some useful abbreviations %%%%%%%%%%%%%%%%
\def\feltor{{\textsc{Feltor }}}

\def\fixme#1{\typeout{FIXME in page \thepage :{#1}}%
 \textsc{\color{red}[{#1}]}}


\usepackage{minted}
\renewcommand{\ne}{\ensuremath{{n_e} }}
\renewcommand{\ni}{\ensuremath{{N_i} }}

%%%%%%%%%%%%%%%%%%%%%%%%%%%%%DOCUMENT%%%%%%%%%%%%%%%%%%%%%%%%%%%%%%%%%%%%%%%
\begin{document}

\title{Diagnostics of geometry}
\author{ M.~Wiesenberger and M.~Held}
\maketitle

\begin{abstract}
  This is a program for 1d, 2d and 3d diagnostics of magnetic field geometry in
  Feltor The purpose of this program is to diagnose geometry.json files with as
  little effort as possible. This program should also remain independent of any
  specific code and therefore does not test or output any initialization
  related functions that require specific parameters in the input file.  We
  currently just write magnetic functions into file and optionally also flux -
  surface averages, the q-profile and sheath coordinates.
\end{abstract}
\tableofcontents

\section{Compilation and useage}
The program geometry\_diag.cpp can be compiled with
\begin{verbatim}
make geometry_diag
\end{verbatim}
Run with
\begin{verbatim}
path/to/geometry_diag/geometry_diag input.json output.nc
\end{verbatim}
The program writes performance informations to std::cout and uses serial netcdf to write a netcdf file.

\section{The input file}
Input file format: \href{https://en.wikipedia.org/wiki/JSON}{json}

%%%%%%%%%%%%%%%%%%%%%%%%%
\subsection{Spatial grid} \label{sec:spatial}
The spatial grid is an equidistant discontinuous Galerkin discretization of the
3D Cylindrical product-space
$[ R_{\min}, R_{\max}]\times [Z_{\min}, Z_{\max}] \times [0,2\pi]$,
where we define
\begin{align} \label{eq:box}
    R_{\min}&=R_0-\varepsilon_{R-}a\quad
    &&R_{\max}=R_0+\varepsilon_{R+}a\nonumber\\
    Z_{\min}&=-\varepsilon_{Z-}a\quad
    &&Z_{\max}=\varepsilon_{Z+}a
\end{align}
We use an equal number of Gaussian nodes in $x$ and $y$.
\begin{minted}[texcomments]{js}
"grid" :
{
    "n"  :  3, // The number of Gaussian nodes in x and y (3 is a good value)
    "Nx"  : 48, // Number of cells in R
    "Ny"  : 48, // Number of cells in Z
    "Nz"  : 20, // Number of cells in varphi
    "scaleR"  : [1.1,1.1], // $[\varepsilon_{R-}, \varepsilon_{R+}]$ scale left and right boundary in R in Eq.\eqref{eq:box}\\
    "scaleZ"  : [1.2,1.1], // $[\varepsilon_{Z-}, \varepsilon_{Z+}]$ scale lower and upper boundary in Z in Eq.\eqref{eq:box}
}
\end{minted}
\begin{minted}[texcomments]{js}
"grid" :
{
    // for flux surface average computations
    "npsi" : 3 // The number of Gaussian nodes in psi and eta
    "Npsi": 64, // resolution of X-point grid for fsa
    "Neta": 640, // resolution of X-point grid for fsa
    "fx_0" : 0.125, // where the separatrix is in relation to the $\zeta$ coordinate
}
\end{minted}
\subsection{Diagnostics}
A user can specify which kind of diagnostics quantities should be computed and
how.
\begin{tcolorbox}[title=Note]
The motivation not to compute all diagnositics is that computing flux
surface averages and sheath coordinates may take a few minutes (as opposed to a
few seconds that the remaining program runs).
Further, there is a risk that
the program fails if the magnetic field parameters are not setup consistently.
\end{tcolorbox}

By default only the standard basic diagnostics is run that has little danger of failing.
ALL of the following can fail and should only be attempted once the parameters are tuned!
The diagnostics field is a list of flags that can be empty
\begin{minted}[texcomments]{js}
"diagnostics" : // Can be empty
[
    "q-profile", // integrate field-lines within LCFS to get q-profile (
            //can fail e.g. if Psi\_p = 0 is not a closed flux-surface)
    "fsa",  //compute a flux-aligned grid and compute flux-surface averaged quantities
    "sheath" // integrate field-lines to get distance to divertor
]
\end{minted}

\subsection{Magnetic field} \label{sec:geometry_file}
The json structure of the geometry parameters depends on which expansion for
$\psi_p$ is chosen. Here, read the documentation of
\href{https://mwiesenberger.github.io/feltor/geometries/html/group__geom.html#gaa0da1d1c2db65f1f4b28d77307ad238b}{`dg::geo::createMagneticField`} and \cite{Cerfon2010}
In addition, we have an option to read the geometry parameters either from an external
file or directly from a field in the input file.
\begin{minted}[texcomments]{js}
"magnetic_field":
{
    // Tells the parser that the geometry parameters are located in an
    // external file the json file (relative to where the program is
    //  executed) containing the geometry parameters to read
    "input": "file",
    "file": "path/to/geometry.json",
    //
    // Tells the parser that the geometry parameters are located in the
    // same file in the params field (recommended option)
    "input": "params",
    "params":
    // copy of the geometry.json
}
\end{minted}
\noindent
\subsection{q-profile}

In the computation of the q-profile using the delta-function method the delta function can have
a width parameter
\begin{minted}[texcomments]{js}
"width-factor" : 0.03
\end{minted}

\subsection{The boundary region} \label{sec:boundary}
Setting the boundary conditions in an appropriate manner is probably the most
fiddly task in setting up a simulation.
%%%%%%%%%%%%%%%%%%%%%%%%%%%%%%%%%%%%%%%%%%%%%%%
\subsubsection{The wall region}\label{sec:wall}
We refer to the documentation of
\href{
https://mwiesenberger.github.io/feltor/geometries/html/group__wall.html#gac47e668c6483d6cebfc9d6c798737b58
}{dg::geo::createModifiedField} for details on how the magnetic field is modified
and how a wall region can be setup.
\begin{minted}[texcomments]{js}
"boundary":
{
    "wall" :
    {
        // no wall region
        "type" : "none"
    }
}
"boundary":
{
    "wall" :
    {
        // Simple flux aligned wall above a threshold value
        "type" : "heaviside",
        "boundary" : 1.2, // wall region boundary $\rho_{p,b}$
        "alpha" : 0.25, // Transition width $\alpha_p$
    }
}
"boundary":
{
    "wall" :
    {
        // Double flux aligned wall above and below a threshold value
        "type" : "sol_pfr",
        "boundary" : [1.2, 0.8],
        "alpha" : [0.25,0.25],
        // first one is for main region, second one for PFR
    }
}
\end{minted}
\noindent

%%%%%%%%%%%%%%%%%%%%%%%%%%%%%%%%%%%%%%%%%%%%%%%%%%%
\subsubsection{The sheath region}\label{sec:sheath}
We refer to the documentation of \href{https://mwiesenberger.github.io/feltor/geometries/html/group__wall.html#gad4fce2a87b624a0d4df4c2431706988f}{dg::geo::createSheathRegion}
and
\href{https://mwiesenberger.github.io/feltor/geometries/html/structdg_1_1geo_1_1_wall_fieldline_coordinate.html}{dg::geo::WallSheathCoordinate}
for details on how we setup the sheath and the Sheath coordinates
\begin{minted}[texcomments]{js}
"boundary":
{
    "sheath" :
    {
        // no sheath region
        "type" : "none",
    }
}
"boundary" :
{
    "sheath" :
    {
        "boundary" : 3/32, // Total width of the sheath $\eps_s$ away from the wall
        // in units of $2\pi$ in Eq.\eqref{eq:sheath}
        "alpha" : 2/32, // Transition width $\alpha$
        // in units of $2\pi$ in Eq.\eqref{eq:sheath}.
        "max_angle" : 4 // $\varphi_{\max}$ in units of $2\pi$
        // in order to compute field-line following coordinates
        // we need to integrate fieldlines. In order to avoid infinite integration
        // we here give a maximum angle where to stop integration
    }
}
\end{minted}
\noindent

\section{Output} \label{sec:output_file}
Output file format: \href{https://www.unidata.ucar.edu/software/netcdf/docs/}{netcdf-4/hdf5};

A \textit{coordinate variable (Coord. Var.)} is a Dataset with the same name as a dimension.
We follow
\href{http://cfconventions.org/Data/cf-conventions/cf-conventions-1.7/cf-conventions.html}{CF Conventions CF-1.7}
and write according attributes into the file.

\begin{tcolorbox}[title=Note]
    The command \mintinline{bash}{ncdump -h output.nc} gives a full list of what a file contains.
\end{tcolorbox}
Here, we list the content without attributes
since the internal netcdf information does not display equations.
%
%Name | Type | Dimensionality | Description
%---|---|---|---|
\begin{longtable}{lll>{\RaggedRight}p{7cm}}
\toprule
\rowcolor{gray!50}\textbf{Name} &  \textbf{Type} & \textbf{Dimension} & \textbf{Description}  \\ \midrule
inputfile  &     text attribute & - & verbose input file as a string (valid JSON, C-style comments are allowed but discarded) \\
x                & Coord. Var. & 1 (x) & $R$-coordinate (computational space, compressed size: $nN_x/c_x$)\\
y                & Coord. Var. & 1 (y) & $Z$-coordinate (computational space, compressed size: $nN_y/c_y$)\\
z                & Coord. Var. & 1 (z) & $\varphi$-coordinate (computational space, size: $N_z$) \\
time             & Coord. Var. & 1 (time)& time at which fields are written (variable size: maxout$+1$, dimension size: unlimited) \\
xc           & Dataset & 3 (z,y,x) & Cartesian x-coordinate $x=R\sin(\varphi)$ \\
yc           & Dataset & 3 (z,y,x) & Cartesian y-coordinate $y=R\cos(\varphi)$\\
zc           & Dataset & 3 (z,y,x) & Cartesian z-coordinate $z=Z$ \\
BR               & Dataset & 3 (z,y,x) & Contravariant magnetic field component $B^R$ \\
BZ               & Dataset & 3 (z,y,x) & Contravariant magnetic field component $B^Z$ \\
BP               & Dataset & 3 (z,y,x) & Contravariant magnetic field component $B^\varphi$ \\
zeta             & Coord. Var. & 1 (psi) & $\psi_p$-coordinate ( default size: $3\cdot 64$) \\
eta              & Coord. Var. & 1 (eta) & $\eta$-coordinate ( default size: $3\cdot 640$) \\
xcc              & Dataset & 2 (eta,zeta) & Cartesian x-coordinate of the FSA grid \\
ycc              & Dataset & 2 (eta,zeta) & Cartesian y-coordinate of the FSA grid\\
time             & Coord. Var. & 1 (time)& time at which fields are written (variable size: maxout$+1$, dimension size: unlimited) \\
dvdpsip          & Dataset & 1 (zeta) & $\d v/\d\psi_p$ \\
psi\_vol         & Dataset & 1 (zeta) & The volume enclosed by the flux surfaces $v(\psi_p) = \int_{\psi_p} \dV $ \\
psi\_area        & Dataset & 1 (zeta) & The area of the flux surfaces $A(\psi_p) = 2\pi \int_\Omega |\vn\psi_p| \delta(\psi_p - \psi_{p0}) H(Z-Z_X) R\d R\d Z$ \\
dvdpsip          & Dataset & 1 (zeta) & $\d v/\d\psi_p$ (in the 2d poloidal plane) \\
q-profile        & Dataset & 1 (zeta) & The safety factor $q(\psi_p)$ \eqref{eq:safety_factor} using direct integration ( accurate but we assign random values outside separatrix) \\
psit1d           & Dataset & 1 (zeta) & Toroidal flux (integrated q-profile) $\psi_t = \int^{\psi_p} \d\psi_p q(\psi_p)$ \\
rho              & Dataset & 1 (zeta) & Transformed flux label $\rho:= 1 - \psi_p/\psi_{p,O}$ \\
rho\_p           & Dataset & 1 (zeta) & poloidal flux label $\rho_p:= \sqrt{1 - \psi_p/\psi_{p,O}}$  ( see Section~\ref{sec:alternative}\\
rho\_t           & Dataset & 1 (zeta) & Toroidal flux label $\rho_t :=
\sqrt{\psi_t/\psi_{t,\mathrm{sep}}}$ (is similar to $\rho$ in the edge but
$\rho_t$ is nicer in the core domain, because equidistant $\rho_t$ make
more equidistant flux-surfaces, see Section~\ref{sec:alternative} )\\
X      & Dataset & 2 (y,x) & 2d version of quantity X \\
X3d    & Dataset & 3 (z,y,x) & 3d version of quantity X \\
X\_fsa & Dataset & 1 (zeta) & flux surface average of quantity X (if "fsa" diagnostics)\\
X\_ifs & Dataset & 1 (zeta) & flux surface integral of quantity X (if "fsa" diagnostics)\\
X\_dfs & Dataset & 1 (zeta) & flux surface derivative of quantity X (if "fsa" diagnostics)\\
\bottomrule
\end{longtable}

The O-point and X-point(s) are written as global attributes "O-point", "X-point" and "2nd X-point" (if exists).

X lies in (search
\href{https://mwiesenberger.github.io/feltor/geometries/html/modules.html}{documentation} for X for further details)
\begin{minted}[texcomments]{cpp}
    {"Psip", "Flux function", mag.psip()},
    {"PsipR", "Flux function derivative in R", mag.psipR()},
    {"PsipZ", "Flux function derivative in Z", mag.psipZ()},
    {"PsipRR", "Flux function derivative in RR", mag.psipRR()},
    {"PsipRZ", "Flux function derivative in RZ", mag.psipRZ()},
    {"PsipZZ", "Flux function derivative in ZZ", mag.psipZZ()},
    {"Ipol", "Poloidal current", mag.ipol()},
    {"IpolR", "Poloidal current derivative in R", mag.ipolR()},
    {"IpolZ", "Poloidal current derivative in Z", mag.ipolZ()},
    {"Rho_p", "Normalized Poloidal flux label", dg::geo::RhoP(mag)},
    {"LaplacePsip", "Laplace of flux function", dg::geo::LaplacePsip(mag)},
    {"Bmodule", "Magnetic field strength", dg::geo::Bmodule(mag)},
    {"InvB", "Inverse of Bmodule", dg::geo::InvB(mag)},
    {"LnB", "Natural logarithm of Bmodule", dg::geo::LnB(mag)},
    {"GradLnB", "The parallel derivative of LnB", dg::geo::GradLnB(mag)},
    {"Divb", "The divergence of the magnetic unit vector", dg::geo::Divb(mag)},
    {"B_R", "Derivative of Bmodule in R", dg::geo::BR(mag)},
    {"B_Z", "Derivative of Bmodule in Z", dg::geo::BZ(mag)},
    {"CurvatureNablaBR",  "R-component of the (toroidal) Nabla B curvature vector", dg::geo::CurvatureNablaBR(mag,+1)},
    {"CurvatureNablaBZ",  "Z-component of the (toroidal) Nabla B curvature vector", dg::geo::CurvatureNablaBZ(mag,+1)},
    {"CurvatureKappaR",   "R-component of the (toroidal) Kappa B curvature vector", dg::geo::CurvatureKappaR(mag,+1)},
    {"CurvatureKappaZ",   "Z-component of the (toroidal) Kappa B curvature vector", dg::geo::CurvatureKappaZ(mag,+1)},
    {"DivCurvatureKappa", "Divergence of the (toroidal) Kappa B curvature vector", dg::geo::DivCurvatureKappa(mag,+1)},
    {"DivCurvatureNablaB","Divergence of the (toroidal) Nabla B curvature vector", dg::geo::DivCurvatureNablaB(mag,+1)},
    {"TrueCurvatureNablaBR", "R-component of the (true) Nabla B curvature vector", dg::geo::TrueCurvatureNablaBR(mag)},
    {"TrueCurvatureNablaBZ", "Z-component of the (true) Nabla B curvature vector", dg::geo::TrueCurvatureNablaBZ(mag)},
    {"TrueCurvatureNablaBP", "Contravariant Phi-component of the (true) Nabla B curvature vector", dg::geo::TrueCurvatureNablaBP(mag)},
    {"TrueCurvatureKappaR", "R-component of the (true) Kappa B curvature vector", dg::geo::TrueCurvatureKappaR(mag)},
    {"TrueCurvatureKappaZ", "Z-component of the (true) Kappa B curvature vector", dg::geo::TrueCurvatureKappaZ(mag)},
    {"TrueCurvatureKappaP", "Contravariant Phi-component of the (true) Kappa B curvature vector", dg::geo::TrueCurvatureKappaP(mag)},
    {"TrueDivCurvatureKappa", "Divergence of the (true) Kappa B curvature vector", dg::geo::TrueDivCurvatureKappa(mag)},
    {"TrueDivCurvatureNablaB","Divergence of the (true) Nabla B curvature vector",  dg::geo::TrueDivCurvatureNablaB(mag)},
    {"BFieldR", "R-component of the magnetic field vector", dg::geo::BFieldR(mag)},
    {"BFieldZ", "Z-component of the magnetic field vector", dg::geo::BFieldZ(mag)},
    {"BFieldP", "Contravariant Phi-component of the magnetic field vector", dg::geo::BFieldP(mag)},
    {"BHatR", "R-component of the magnetic field unit vector", dg::geo::BHatR(mag)},
    {"BHatZ", "Z-component of the magnetic field unit vector", dg::geo::BHatZ(mag)},
    {"BHatP", "Contravariant Phi-component of the magnetic field unit vector", dg::geo::BHatP(mag)},
    {"BHatRR", "R derivative of BHatR", dg::geo::BHatRR(mag)},
    {"BHatZR", "R derivative of BHatZ", dg::geo::BHatZR(mag)},
    {"BHatPR", "R derivative of BHatP", dg::geo::BHatPR(mag)},
    {"BHatRZ", "Z derivative of BHatR", dg::geo::BHatRZ(mag)},
    {"BHatZZ", "Z derivative of BHatZ", dg::geo::BHatZZ(mag)},
    {"BHatPZ", "Z derivative of BHatP", dg::geo::BHatPZ(mag)},
    {"NormGradPsip", "Norm of gradient of Psip", dg::geo::SquareNorm( dg::geo::createGradPsip(mag), dg::geo::createGradPsip(mag))},
    {"SquareGradPsip", "Norm of gradient of Psip", dg::geo::ScalarProduct( dg::geo::createGradPsip(mag), dg::geo::createGradPsip(mag))},
    {"CurvatureNablaBGradPsip", "(Toroidal) Nabla B curvature dot the gradient of Psip", dg::geo::ScalarProduct( dg::geo::createCurvatureNablaB(mag, +1), dg::geo::createGradPsip(mag))},
    {"CurvatureKappaGradPsip", "(Toroidal) Kappa curvature dot the gradient of Psip", dg::geo::ScalarProduct( dg::geo::createCurvatureKappa(mag, +1), dg::geo::createGradPsip(mag))},
    {"TrueCurvatureNablaBGradPsip", "True Nabla B curvature dot the gradient of Psip", dg::geo::ScalarProduct( dg::geo::createTrueCurvatureNablaB(mag), dg::geo::createGradPsip(mag))},
    {"TrueCurvatureKappaGradPsip", "True Kappa curvature dot the gradient of Psip", dg::geo::ScalarProduct( dg::geo::createTrueCurvatureKappa(mag), dg::geo::createGradPsip(mag))},
    //////////////////////////////////
    {"Iris", "A flux aligned Iris", dg::compose( dg::Iris( 0.5, 0.7), dg::geo::RhoP(mag))},
    {"Pupil", "A flux aligned Pupil", dg::compose( dg::Pupil(0.7), dg::geo::RhoP(mag)) },
    {"PsiLimiter", "A flux aligned Heaviside", dg::compose( dg::Heaviside( 1.03), dg::geo::RhoP(mag) )},
    {"MagneticTransition", "The region where the magnetic field is modified", transition},
    {"Delta", "A flux aligned Gaussian peak", dg::compose( dg::GaussianX( psipO*0.2, deltaPsi, 1./(sqrt(2.*M_PI)*deltaPsi)), mag.psip())},
    ////
    { "Hoo", "The novel h02 factor", dg::geo::Hoo( mag) },
    {"Wall", "Penalization region that acts as the wall", wall },
    {"WallDistance", "Distance to closest wall", dg::geo::CylindricalFunctor( dg::WallDistance( sheath_walls)) }
\end{minted}

If the "sheath" diagnostics is activated then additionally the following fields are computed
\begin{minted}{cpp}
    {"WallFieldlineAnglePDistance", "Distance to wall along fieldline",
        dg::geo::WallFieldlineDistance( dg::geo::createBHat(mod_mag),
                sheath_walls, maxPhi, 1e-6, "phi") },
    {"WallFieldlineAngleMDistance", "Distance to wall along fieldline",
        dg::geo::WallFieldlineDistance( dg::geo::createBHat(mod_mag),
                sheath_walls, -maxPhi, 1e-6, "phi") },
    {"WallFieldlineSPDistance", "Distance to wall along fieldline",
        dg::geo::WallFieldlineDistance( dg::geo::createBHat(mod_mag),
                sheath_walls, maxPhi, 1e-6, "s") },
    {"WallFieldlineSMDistance", "Distance to wall along fieldline",
        dg::geo::WallFieldlineDistance( dg::geo::createBHat(mod_mag),
                sheath_walls, -maxPhi, 1e-6, "s") },
    {"Sheath", "Sheath region", sheath},
    {"SheathDirection", "Direction of magnetic field relative to sheath", dg::geo::WallDirection(mag, sheath_walls) },
    {"SheathCoordinate", "Coordinate from -1 to 1 of magnetic field relative to sheath", dg::geo::WallFieldlineCoordinate( dg::geo::createBHat( mod_mag), sheath_walls, maxPhi, 1e-6, "s")}
\end{minted}

%%%%%%%%%%%%%%%%%%%%%%%%%%%%%%%%

%..................................................................
\bibliography{../common/references}
%..................................................................

\end{document}

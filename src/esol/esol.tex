%%%%%%%%%%%%%%%%%%%%%definitions%%%%%%%%%%%%%%%%%%%%%%%%%%%%%%%%%%%%%%%
%\documentclass[12pt]{article}
%\documentclass[12pt]{scrartcl}
\documentclass{hitec} % contained in texlive-latex-extra
\settextfraction{0.9} % indent text
\usepackage{csquotes}
\usepackage[hidelinks]{hyperref} % doi links are short and usefull?
\hypersetup{%
    colorlinks=true,
    linkcolor=blue,
    urlcolor=magenta
}
\urlstyle{rm}
\usepackage[english]{babel}
\usepackage{mathtools} % loads and extends amsmath
\usepackage{amssymb}
% packages not used
%\usepackage{graphicx}
%\usepackage{amsthm}
%\usepackage{subfig}
\usepackage{bm}
\usepackage{geometry}
\usepackage{longtable}
\usepackage{booktabs}
\usepackage{ragged2e} % maybe use \RaggedRight for tables and literature?
\usepackage[table]{xcolor} % for alternating colors
%\rowcolors{2}{gray!25}{white} %%% Use this line in front of longtable
\renewcommand\arraystretch{1.3}
\usepackage[most]{tcolorbox}
\usepackage{doi}
\usepackage[sort,square,numbers]{natbib}
\bibliographystyle{abbrvnat}
%%% reset bibliography distances %%%
\let\oldthebibliography\thebibliography
\let\endoldthebibliography\endthebibliography
\renewenvironment{thebibliography}[1]{
  \begin{oldthebibliography}{#1}
    \RaggedRight % remove if justification is desired
    \setlength{\itemsep}{0em}
    \setlength{\parskip}{0em}
}
{
  \end{oldthebibliography}
}
%%% --- %%%

\definecolor{light-gray}{gray}{0.95}
\newcommand{\code}[1]{\colorbox{light-gray}{\texttt{#1}}}
\newcommand{\eps}{\varepsilon}
\renewcommand{\d}{\mathrm{d}}
\renewcommand{\vec}[1]{{\boldsymbol{#1}}}
\newcommand{\dx}{\,\mathrm{d}x}
%\newcommand{\dA}{\,\mathrm{d}(x,y)}
%\newcommand{\dV}{\mathrm{d}^3{x}\,}
\newcommand{\dA}{\,\mathrm{dA}}
\newcommand{\dV}{\mathrm{dV}\,}

\newcommand{\Eins}{\mathbf{1}}

\newcommand{\ExB}{$\bm{E}\times\bm{B} \,$}
\newcommand{\GKI}{\int d^6 \bm{Z} \BSP}
\newcommand{\GKIV}{\int dv_{\|} d \mu d \theta \BSP}
\newcommand{\BSP}{B_{\|}^*}
\newcommand{\Abar}{\langle A_\parallel \rangle}
%Averages
\newcommand{\RA}[1]{\left \langle #1 \right \rangle} %Reynolds (flux-surface) average
\newcommand{\RF}[1]{\widetilde{#1}} %Reynolds fluctuation
\newcommand{\FA}[1]{\left[\left[ #1 \right]\right]} %Favre average
\newcommand{\FF}[1]{\widehat{#1}} %Favre fluctuation
\newcommand{\PA}[1]{\left \langle #1 \right\rangle_\varphi} %Phi average

%Vectors
\newcommand{\ahat}{\bm{\hat{a}}}
\newcommand{\bhat}{\bm{\hat{b}}}
\newcommand{\chat}{\bm{\hat{c}}}
\newcommand{\ehat}{\bm{\hat{e}}}
\newcommand{\bbar}{\overline{\bm{b}}}
\newcommand{\xhat}{\bm{\hat{x}}}
\newcommand{\yhat}{\bm{\hat{y}}}
\newcommand{\zhat}{\bm{\hat{z}}}

\newcommand{\Xbar}{\bar{\vec{X}}}
\newcommand{\phat}{\bm{\hat{\perp}}}
\newcommand{\that}{\bm{\hat{\theta}}}

\newcommand{\eI}{\bm{\hat{e}}_1}
\newcommand{\eII}{\bm{\hat{e}}_2}
\newcommand{\ud}{\mathrm{d}}

%Derivatives etc.
\newcommand{\pfrac}[2]{\frac{\partial#1}{\partial#2}}
\newcommand{\ffrac}[2]{\frac{\delta#1}{\delta#2}}
\newcommand{\fixd}[1]{\Big{\arrowvert}_{#1}}
\newcommand{\curl}[1]{\nabla \times #1}

\newcommand{\np}{\vec{\nabla}_{\perp}}
\newcommand{\npc}{\nabla_{\perp} \cdot }
\newcommand{\nc}{\vec\nabla\cdot}
\newcommand{\cn}{\cdot\vec\nabla}
\newcommand{\vn}{\vec{\nabla}}
\newcommand{\npar}{\nabla_\parallel}

\newcommand{\KB}{ {\vec K_{\vn B}}}
\newcommand{\KK}{ {\vec K_{\vn\times\bhat}}}

\newcommand{\GAI}{\Gamma_{1}^{\dagger}}
\newcommand{\GAII}{\Gamma_{1}^{\dagger -1}}
\newcommand{\T}{\mathrm{T}}
\newcommand{\Tp}{\mathcal T^+_{\Delta\varphi}}
\newcommand{\Tm}{\mathcal T^-_{\Delta\varphi}}
\newcommand{\Tpm}{\mathcal T^\pm_{\Delta\varphi}}
\newcommand{\Tdp}{\mathcal T^+_{\delta\varphi}}
\newcommand{\Tdm}{\mathcal T^-_{\delta\varphi}}
\newcommand{\Tdpm}{\mathcal T^\pm_{\delta\varphi}}
%%%%%%%%Some useful abbreviations %%%%%%%%%%%%%%%%
\def\feltor{{\textsc{Feltor }}}

\def\fixme#1{\typeout{FIXME in page \thepage :{#1}}%
 \textsc{\color{red}[{#1}]}}



\usepackage{minted}

%%%%%%%%%%%%%%%%%%%%%%%%%%%%%DOCUMENT%%%%%%%%%%%%%%%%%%%%%%%%%%%%%%%%%%%%%%%
\begin{document}

\title{The esol project}
\author{M.~Held}
\maketitle

\begin{abstract}
  This is a program for 2d isothermal full-f gyro-fluid turbulence simulations 
that cover the edge and scrape off layer and include different treatments of the 
polarization charge
\end{abstract}

\section{Equations}
Currently we implemented $4$ slightly different sets of equations. $n$ is the 
electron density, $N$ is the ion gyrocentre density. $\phi$ is the electric 
potential. We
use Cartesian coordinates $x$, $y$.
\subsection{Models}
\subsubsection{Full-F gyro-fluid models}
The full-f models ("ff-O2-OB" \& "ff-lwl-OB" \& "ff-lwl" \& "ff-O2") share the 
same evolution equations
\begin{subequations}
\begin{align}
 \frac{\partial n}{\partial t}     &= 
    \frac{1}{B}\{ n, \phi\} 
  + \kappa n\frac{\partial \phi}{\partial y} 
  -\kappa \frac{\partial n}{\partial y} + \Lambda_{n,\parallel}
  - \nu \Delta_\perp^2 n  \\
  \frac{\partial N}{\partial t} &=
  \frac{1}{B}\{ N, \psi\} 
  + \kappa N\frac{\partial \psi}{\partial y} 
  + \tau \kappa\frac{\partial N}{\partial y} + \Lambda_{N,\parallel}-\nu 
\Delta_\perp^2 N \\
   \vec{\nabla}\cdot \vec{P}_2 &= \Gamma_1 N-n
\end{align}
\end{subequations}
and are based upon a ``radially'' varying magnetic field magnitude
\begin{align}
 B(x)^{-1} = \kappa x +1-\kappa X 
\end{align}
The full-f models differ only in the treatment the  polarization terms 
\(\psi_2\) and \(\vec{P}_2\), given in Sec.~\ref{sec:polarization}
%%%%%
\subsubsection{Polarization and FLR terms}
\label{sec:polarization}
The FLR and polarization operators enter the gyro-fluid potential (\(\psi)\) as 
well as the polarization density.
The operators and gyro-fluid potential are
\begin{align}
 \Gamma_1 &= ( 1- \tau/2 \Delta_{\perp})^{-1} &
 \Gamma_0 &= ( 1- \tau\Delta_{\perp})^{-1} &
   \psi = \Gamma_1 \phi + \psi_2 
\end{align}
The  (negative) polarization charge is~\cite{Held2020} 
\begin{align}
  \vec{\nabla}\cdot \vec{P}_2 &=    
\begin{cases}
-\frac{N_0}{B_0^2}\Delta_{\perp} \phi , 
        &\ \text{if "ff-lwl-OB"} \\
-\frac{N_0}{B_0^2}\Gamma_0\Delta_{\perp} \phi, 
      &\ \text{if "ff-O2-OB"} \\
-\nabla\cdot \left(\frac{N}{B^2} \nabla_\perp \phi\right)
       &\ \text{if "ff-lwl"} \\
-\sqrt{\Gamma_0}\nabla\cdot \left(\frac{N}{B^2} \nabla_\perp\sqrt{\Gamma_0} 
\phi\right) 
         &\ \text{if "ff-O2"} 
\end{cases}
\end{align}
The polarization part of the gyro-fluid potential is
\begin{align}
\psi_2&=    
\begin{cases}
0 , 
        &\ \text{if "ff-lwl-OB"} \\
0, 
      &\ \text{if "ff-O2-OB"} \\
- \frac{1}{2} \frac{(\nabla\phi)^2}{B^2}
       &\ \text{if "ff-lwl"} \\
- \frac{1}{2} \frac{(\nabla\sqrt{\Gamma_0}\phi)^2}{B^2}
         &\ \text{if "ff-O2"} 
\end{cases}
\end{align}
\subsubsection{Closure of the parallel dynamics}
\label{sec:parallelclosure}
The closure terms for the parallel dynamics exploit a two region 
approach~\cite{HeldPhD}: a closed field line and an open field line region. The 
transition between this regions is mimiced by the damping function \( 
h_{\pm}(x)\)
where \(h_{+}\) is a polynomial Heaviside function (cf. feltor docu) and the 
\(-\) refers to its reflection in x-direction. The polynomial Heaviside function 
parameters are \(x_s\) and \(\sigma_s\).
Here, \(x_s\) represents the position of the separatrix and \(\sigma_s\) is the 
width of the transition. 
\\
The complete parallel closure consists then of two contributions:
\begin{align}
 \Lambda_{n,\parallel} &= \sum_{b \in \left\{+,-\right\} 
}\Lambda_{n,\parallel,b} &
 \Lambda_{N,\parallel} &= \sum_{b \in \left\{+,-\right\} } 
\Lambda_{N,\parallel,b}
\end{align}

\paragraph{Edge:} In the edge region we take the modified Hasegawa-Waktani ("modified")
closure~\cite{Held2018}
\begin{align}
 \Lambda_{n,\parallel,-} &= \alpha h_{-}(x) \left[\widetilde{\phi} - 
\widetilde{\ln(n)} \right] \\
 \Lambda_{N,\parallel,-} &= 0
\end{align}
with the fluctuation given by \(\widetilde{f} := f - \langle f\rangle_y\) and 
the y-average \(\langle f\rangle_y:= \int dy f / L_y\). The adiabaticity 
parameter is defined as \(\alpha := T_{e,\parallel} 
k_\parallel^2/(\eta_\parallel e^2 n_{e,0} \Omega_{i,0} )\).
\\
Alternatively, the ordinary Hasegawa-Wakatani ("ordinary") closure is given by~\cite{Held2018}
\begin{align}
 \Lambda_{n,\parallel,-} &= \alpha h_{-}(x) \left[\phi - 
\ln(n/\langle n\rangle) \right] \\
 \Lambda_{N,\parallel,-} &= 0
\end{align}
The third option is to normalize the logarithmic term to a constant reference density ("ordinary\_nonper")
\begin{align}
 \Lambda_{n,\parallel,-} &= \alpha h_{-}(x) \left[\phi - 
\ln(n/n_0) \right] \\
 \Lambda_{N,\parallel,-} &= 0
\end{align}
\paragraph{Scrape-off-Layer:} In the scrape off layer we assume Bohm sheath 
boundary conditions~\cite{Mosetto2015}~\footnote{Note that strictly speaking the 
boundary conditions for the density and electric potential are Neumann, in 
particular \(\partial_s n_e (\vec{x}) |_{\pm L_\parallel /2 } =0 \) and 
\(\partial_s  n_e (\vec{x}) |_{\pm L_\parallel /2 } = 0\).}
\begin{align}
n_e (x,y,\pm L_\parallel /2 ) &= n_e (\vec{x}) &
\phi (x,y,\pm L_\parallel /2 ) &= \phi (\vec{x})\\
 u_e (x,y,\pm L_\parallel /2 ) &= \pm \exp{(\Lambda_{sh} - \phi)} &
 U_i (x,y,\pm L_\parallel /2 ) &= \pm\sqrt{1+\tau}
\end{align}
where we introduced the constant (also known as sheath or Bohm potential) 
\(\Lambda_{sh} := \ln \sqrt{m_i/(2 \pi m_e)} = \ln  (1/\sqrt{|\mu_e| 2 \pi})\). 
With this we obtain for the closure terms~\cite{HeldPhD}
\begin{align}
 \Lambda_{n,\parallel,+} &:= -h_{+}(x) n \lambda \exp{(\Lambda_{sh} - \phi)} \\
 \Lambda_{N,\parallel,+} &:=  -\sqrt{1 + \tau} \lambda \Gamma_1^{-1} ( h_{+}(x)  
n)
\end{align}
with the so called sheath dissipation parameter \(\lambda = 
\rho_s/L_\parallel\).
Note that this results into the following term in the vorticity density equation
\begin{align}
 \Lambda_{\mathcal{W},\parallel,+} := \Gamma_1\Lambda_{N,\parallel,+} - 
\Lambda_{n,\parallel,+} = -\lambda  h_{+}(x)  n \sqrt{1 + \tau} \left[1  -  
\frac{1}{\sqrt{1 + \tau}}\exp{(\Lambda_{sh} - \phi)} \right]
\end{align}
In order to circumvent \ExB shear flows on the boundaries when we choose Dirichlet boundary conditions we can renormalize our potential. We choose
\begin{align}
 \phi' := \phi - \Lambda_{sh}+ \ln{\sqrt{1+\tau}}
\end{align}
and substitute into the relevant closure term
\begin{align}
 \Lambda_{n,\parallel,+} &:= -\sqrt{1 + \tau}h_{+}(x) n \lambda \exp{(- \phi')} 
\end{align}
to obtain
\begin{align}
 \Lambda_{\mathcal{W},\parallel,+} :=  -\lambda  h_{+}(x)  n \sqrt{1 + \tau} \left[1  -  
\exp{(-\phi')} \right]
\end{align}
which is vanishing for \(\phi'=0\) at the boundaries. Note that the density and vorticity density sink rate increases with \(\tau\)!
\subsubsection{Sources}
The particle source is related to the the ion gyro-center source 
and the electric potential via the transformation rule
\begin{align}
 S_n =\Gamma_1 S_N +
 \begin{cases}
0, 
        &\ \text{if "ff-lwl-OB"} \\
0, 
      &\ \text{if "ff-O2-OB"} \\
\nabla\cdot \left(\frac{S_N}{B^2} \nabla_\perp \phi\right)
       &\ \text{if "ff-lwl"} \\
\sqrt{\Gamma_0}\nabla\cdot \left(\frac{S_N}{B^2} \nabla_\perp\sqrt{\Gamma_0} 
\phi\right) 
         &\ \text{if "ff-O2"} 
\end{cases}
\end{align}
Instead, we could use the long wavelength approximation \(
S_N =(1+\tau/2 \Delta_\perp) S_n + \vec{\nabla} \cdot \left(\frac{S_n}{B^2} 
\vec{\nabla}_\perp \phi\right)\). However this transformation rule is not exact 
and care has to be taken when short wavelength structures arise! \\
Another possibility is to neglect the polarization charge term in the exact transformation rule, which results in a vanishing source term in the evolution equation of the polarization charge density.
\paragraph{Forced profile (``forced'')}
The following source forces the poloidally averaged profile to a prescribed profile in a limited region if the averaged profile is below the prescribed profile. 
\begin{align}
 S_N := \omega_s p \Theta\left(p \right)
\end{align}
with \(p = h_s (n_{prof,s} - \langle N \rangle_y)\) with \(h_s\) a polynomial heaviside function.

\paragraph{Constant influx (``flux'')}
\begin{align}
 S_N := \omega_s n_{prof,s} 
\end{align}
with the source profile function 
\begin{align}
 n_{prof,s}&:= 
 \begin{cases}
   e^{1+\left(\frac{(x-x_s)^2}{2 \sigma_s^2}-1\right)^{-1}}, &\frac{(x-x_s)^2}{2 
\sigma_s^2}<1  \\
   0, & else
 \end{cases}
\end{align}
with \(x_s:= l_x  f_s\)
\subsection{Initialization}
We follow the strategy to enforce the initial fields of the physical variables, 
the electron density \(n\) and the electric potential \(\phi\), in order to 
compute the initial ion gyro-center density.
\subsubsection{Non-rotating Gaussian ("blob")}
Initialization of $n$ is a Gaussian 
\begin{align}
    n(x,y,0) &= 1 + A\exp\left( -\frac{(x-X)^2 + (y-Y)^2}{2\sigma^2}\right) \\
    \phi(x,y)&=const.
\end{align}
where $X = p_x l_x$ and $Y=p_yl_y$ are the initial centre of mass position 
coordinates, $A$ is the amplitude and $\sigma$ the
radius of the blob.
We initialize 
\begin{align}
    N &= \Gamma_1^{-1} n 
\end{align}
\subsubsection{Gaussian with zero polarization charge density("blob\_zeropol")}
Initialization of $n$ is a Gaussian 
\begin{align}
    n(x,y,0) &= 1 + A\exp\left( -\frac{(x-X)^2 + (y-Y)^2}{2\sigma^2}\right) \\
\end{align}
where $X = p_x l_x$ and $Y=p_yl_y$ are the initial centre of mass position 
coordinates, $A$ is the amplitude and $\sigma$ the radius of the blob. We 
initialize then
\begin{align}
    N &= n 
\end{align}
so that the total polarization charge vanishes  \(\vec{\nabla}\cdot \vec{P}=0\).
\subsection{Diagnostics}
Diagnostics are the mass \(M\)
\begin{align}
    M(t) &:= \int dA (n)  \\
\end{align}
\section{Numerical methods}
discontinuous Galerkin on structured grid
\rowcolors{2}{gray!25}{white} %%% Use this line in front of longtable
\begin{longtable}{ll>{\RaggedRight}p{7cm}}
\toprule
\rowcolor{gray!50}\textbf{Term} &  \textbf{Method} & \textbf{Description}  \\ 
\midrule
coordinate system & Cartesian 2D & equidistant discretization of $[0,l_x] \times 
[0,l_y]$, equal number of Gaussian nodes in x and y \\
matrix inversions & multigrid conjugate gradient &  \\
matrix functions & cauchy integral  method & \\
\ExB advection & centered upwind-scheme\\
curvature terms & centered difference & \\
time &  adaptive explicit RK or explicit multistep  &  \\
\bottomrule
\end{longtable}
\section{Compilation and useage}
There are two programs esol.cu and esol\_hpc.cu . Compilation with
\begin{verbatim}
make <esol esol_hpc esol_mpi> device = <omp gpu>
\end{verbatim}
Run with
\begin{verbatim}
path/to/feltor/src/esol/esol input.json
path/to/feltor/src/esol/esol_hpc input.json output.nc
echo np_x np_y | mpirun -n np_x*np_y path/to/feltor/src/esol/esol_mpi/
    input.json output.nc
\end{verbatim}
All programs write performance informations to std::cout.
The first is for shared memory systems (OpenMP/GPU) and opens a terminal window 
with life simulation results.
 The
second can be compiled for both shared and distributed memory systems and uses 
serial netcdf in both cases
to write results to a file.
For distributed
memory systems (MPI+OpenMP/GPU) the program expects the distribution of 
processes in the
x and y directions as command line input parameters.

\subsection{Input file structure}
Input file format: json
\begin{minted}[texcomments]{js}
"grid":
{
    "n" :  5,    //Legendre polynomial order in x and y 
    "Nx" : 32,   //grid points in x
    "Ny" : 32,   //grid points in y
    "lx"  : 64,  //Box size in x in units of $\rho_s$
    "ly"  : 64   //Box size in y in units of $\rho_s$
},
"timestepper":
{
    "type": "adaptive", //"adaptive" (explicit adaptive RK) or "multistep" (explicit multistep)
    "tableau": "Bogacki-Shampine-4-2-3", // recommended "Bogacki-Shampine-4-2-3" (default adaptive) or "TVB-3-3" (multistep)
    "rtol": 1e-5, //relative tolerance of adaptive time stepper
    "atol": 1e-7, //absolute tolerance of adaptive time stepper
    "dt" : 0.1   //time step in units of $c_s/\rho_s$ for multistep. Also determines together with itstep output step $dt_{out} = dt itstp$
},
"output":
{
    "type": "glfw",  // output format "glfw" & "netcdf",
    "itstp"  : 1,    //time steps between outputs
    "maxout" : 2,    //\# of netcdf outputs
    "n" : 5,         //Legendre polynomial order in x and y for netcdf output
    "Nx" : 32,       //grid points in x for netcdf output
    "Ny" : 32        //grid points in y for netcdf output
},    
"elliptic":
{
    "stages"     : 3, //\# of stages of the multigrid solver for the (tensor) elliptic operator
    "eps_pol"    : [1e-6,1.0,1.0], //accuracy at each stage of the multigrid solver
    "jumpfactor": 1 //jump factor of the (tensor) elliptic operator
},
"helmholtz":
{
    "eps_gamma1" :   1e-8, //accuracy of the $\Gamma_1$ operator
    "eps_gamma0" :   1e-6, //accuracy of the $\Gamma_0$ or $\sqrt{\Gamma_0}$ operator
    "maxiter_sqrt":  200,  //max iterations of the $\sqrt{\Gamma_0}$ computation
    "maxiter_cauchy": 35,  //max iterations of the Cauchy terms in $\sqrt{\Gamma_0}$ computation
    "eps_cauchy" :  1e-12  //accuracy of the Cauchy integral in the $\sqrt{\Gamma_0}$ computation
},

"physical":
{
    "curvature"  : 0.00015, //$\kappa$
    "tau"  : 4.0,           //$\tau_i = T_{i0}/(T_{e0} Z_i)$
    "alpha"  : 0.005,       //adiabaticity $\alpha:= T_{e,\parallel} k_\parallel^2/(\eta_\parallel e^2 n_{e,0} \Omega_{i,0} )$
    "hwmode" : "modified"   //hasegawa-wakatani model $\in$ ("modified" ,"ordinary" ,"ordinary\_nonper" )
    "lambda"  : 0.000001,    //sheath dissipation $\lambda:= \rho_{s0}/L_\parallel$
    "mu_e"  : -0.000272121, //$\mu_e:=m_e/(Z_e m_i)$ negative electron to ion mass ratio
    "equations": "ff-O2",   //model $\in$ ("ff-lwl-OB" ,"ff-O2-OB" ,"ff-lwl" ,"ff-O2" ,"ff-O4" )
    "renormalize": true,    //Renormalize potential (useful for DIR conditions with finite sheath dissipation)
    "xfac_sep"  : 0.3,      //$f_s$ x-position of separatrix in units of $l_x$     
    "sigma_sep"  : 0.5      //$\sigma_s$ width of damping function $h_{pm}$ of separatrix in units of $\rho_{s0}$    
},
"source":
{
     "source_shape" :"cauchy",  //type of profile $\in$ ("cauchy", "gaussian")
     "source_type" : "flux",    //type of source $\in$ ("forced", "flux")
     "source_rel" : "zero_pol", //relation for source, $\in$ ("zero-pol", "finite-pol")
     "omega_s" : 0.005,         //source rate in units of $\Omega_i$
     "xfac_s": 0.1,             //position of source in units of the $l_x$
     "sigma_s" : 5.0,           //width of source in units of $\rho_{s0}$
     "n_min" :1e-3,             //minimum density that is enforced
     "omega_n": 1e8             //forcing rate for minimum density 
},
"profile":
{
     "bgproftype": "exp",  //type of background profile $\in$ ("exp","tanh")
     "bgprofamp" : 1.0, //amplitude of background in units of $n_{e0}$, typically 1
     "profamp": 0.0,    //amplitude of profile in units of $n_{e0}$
     "ln": 20.0,    //width of gradient region of the profile in units of $\rho_{s0}$
     "xfac_p": 0.5   //position of tanh profile in units of box size in x-direction
},
"init":
{
    "type"       : "blob",  // Gaussian blob initialization
    "amplitude"  :1.0,      //$A$ of the Gaussian blob
    "mx": 1.0,              //Wavenumber in x-direction
    "my":1.0,               //Wavenumber in y-direction
    "sigma"  : 5,           //$\sigma$ of Gaussian blob units of $\rho_s$
    "posX"  : 0.5,          //x initial position $\in (0,1)$
    "posY"  : 0.5           //y initial position $\in (0,1)$
    "xfac_d": 0.05,         //damping position of initial perturbation at left and right boundaries in x
    "sigma_d" : 4.0         //damping transition width
},
"nu_perp"  : 0e-5,          //hyper-diffusion parameter
"bc_x"  : "DIR_NEU",        //Boundary condtion in x for $\phi$ and $\psi$
"bc_N_x": "NEU",            //Boundary condition in x for the densities
"bc_y"  : "PER",             //Boundary condtion in y
"formulation": "conservative"//formulation of the continuity equations $\in ("conservative","ln")$
\end{minted}
\subsection{Structure of output file}
Output file format: netcdf-4/hdf5
%
%Name | Type | Dimensionality | Description
%---|---|---|---|
\begin{longtable}{lll>{\RaggedRight}p{7cm}}
\toprule
\rowcolor{gray!50}\textbf{Name} & 
 \textbf{Type} & \textbf{Dimension} & \textbf{Description}  \\ \midrule
inputfile  &             text attribute & 1 & verbose input file as a string \\
energy\_time             & Dataset & 1 & timesteps at which 1d variables are 
written \\
time                     & Dataset & 1 & time at which fields are written \\
x                        & Dataset & 1 & x-coordinate  \\
y                        & Dataset & 1 & y-coordinate \\
electrons                & Dataset & 3 (time, y, x) & electon density $n$ \\
ions                     & Dataset & 3 (time, y, x) & ion density $N$ \\
potential                & Dataset & 3 (time, y, x) & electric potential $\phi$  
\\
vorticity                & Dataset & 3 (time, y, x) & z-component of ExB 
vorticity  $\Omega_E = \vec{\nabla}\cdot (B^{-1} \vec{\nabla}_{\perp}\phi)$  \\
lperpinv                 & Dataset & 3 (time, y, x) & inverse perp density gradient length scale $L_\perp^{-1} := |\nabla_\perp n| / n$ \\
lperpinvphi                 & Dataset & 3 (time, y, x) & inverse perp eletric potential gradient length scale $L_{\perp,\phi}^{-1} := |\nabla_\perp \phi| $ \\
% dEdt                     & Dataset & 1 (energy\_time) & change of energy per 
% dissipation              & Dataset & 1 (energy\_time) & diffusion integrals  
\\
% energy                   & Dataset & 1 (energy\_time) & total energy integral  
\\
% mass                     & Dataset & 1 (energy\_time) & mass integral   \\
\bottomrule
\end{longtable}
%..................................................................
\bibliography{../common/references}
%..................................................................
\bibliographystyle{aipnum4-1.bst}

\end{document}
